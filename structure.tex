%%%%%%%%%%%%%%%%%%%%%%%%%%%%%%%%%%%%%%%%%
% Wenneker Resume/CV
% Structure Specification File
% Version 1.1 (19/6/2016)
%
% This file has been downloaded from:
% http://www.LaTeXTemplates.com
%
% Original author:
% Frits Wenneker (http://www.howtotex.com) with extensive modifications by 
% Vel (vel@latextemplates.com)
%
% License:
% CC BY-NC-SA 3.0 (http://creativecommons.org/licenses/by-nc-sa/3.0/)
%
%%%%%%%%%%%%%%%%%%%%%%%%%%%%%%%%%%%%%%%%%

%----------------------------------------------------------------------------------------
%	PACKAGES AND OTHER DOCUMENT CONFIGURATIONS
%----------------------------------------------------------------------------------------

\usepackage{XCharter} % Use the Bitstream Charter font
\usepackage[utf8]{inputenc} % Required for inputting international characters
\usepackage[T1]{fontenc} % Output font encoding for international characters

\usepackage[top=1cm,left=1cm,right=1cm,bottom=1cm]{geometry} % Modify margins

\usepackage{graphicx} % Required for figures

\usepackage{flowfram} % Required for the multi-column layout

\usepackage{url} % URLs

\usepackage[usenames,dvipsnames]{xcolor} % Required for custom colours

\usepackage{tikz} % Required for the horizontal rule

\usepackage{enumitem} % Required for modifying lists
\setlist{noitemsep,nolistsep} % Remove spacing within and around lists

\setlength{\columnsep}{\baselineskip} % Set the spacing between columns

% Define the left frame (sidebar)
\newflowframe{0.2\textwidth}{\textheight}{0pt}{0pt}[left]
\newlength{\LeftMainSep}
\setlength{\LeftMainSep}{0.2\textwidth}
\addtolength{\LeftMainSep}{1\columnsep}
 
% Small static frame for the vertical line
\newstaticframe{1.5pt}{\textheight}{\LeftMainSep}{0pt}
 
% Content of the static frame with the vertical line
\begin{staticcontents}{1}
\hfill
\tikz{\draw[loosely dotted,color=RoyalBlue,line width=1.5pt,yshift=0](0,0) -- (0,\textheight);}
\hfill\mbox{}
\end{staticcontents}
 
% Define the right frame (main body)
\addtolength{\LeftMainSep}{1.5pt}
\addtolength{\LeftMainSep}{1\columnsep}
\newflowframe{0.7\textwidth}{\textheight}{\LeftMainSep}{0pt}[main01]

\pagestyle{empty} % Disable all page numbering

\setlength{\parindent}{0pt} % Stop paragraph indentation

%----------------------------------------------------------------------------------------
%	NEW COMMANDS
%----------------------------------------------------------------------------------------

\newcommand{\CVDate}[0]{ % New command for CV datetime info
\begin{flushright}
	\textit{Updated: \today}
\end{flushright}
}
\newcommand{\userinformation}[1]{\renewcommand{\userinformation}{#1}} % Define a new command for the CV user's information that goes into the left column

\newcommand{\cvheading}[1]{{\Huge\bfseries\color{RoyalBlue} #1} \par\vspace{.6\baselineskip}} % New command for the CV heading
\newcommand{\cvsubheading}[1]{{\Large\bfseries #1} \bigbreak} % New command for the CV subheading

\newcommand{\Sep}{\vspace{1em}} % New command for the spacing between headings
\newcommand{\SmallSep}{\vspace{0.5em}} % New command for the spacing within headings

\newcommand{\aboutme}[2]{ % New command for the about me section
\textbf{\color{RoyalBlue} #1}~~#2\par\Sep
}
	
\newcommand{\CVSection}[1]{ % New command for the headings within sections
{\Large\textbf{#1}}\par
\SmallSep % Used for spacing
}

\newcommand{\CVItem}[2]{ % New command for the item descriptions
\textbf{\color{RoyalBlue} #1}\par
#2
\SmallSep % Used for spacing
}

\newcommand{\CVRecord}[5]{ % New command for a data record
\textbf{#1}\hfill\textit{#5}\par
\textbf{\color{RoyalBlue}\textit{#2}}\hfill\textbf{\color{RoyalBlue} #3}\par
#4
\Sep % Used for spacing
}

\newcommand{\CVReference}[4]{ % New command for references
\textbf{#1}\par
#2, #3
\par
Email: #4
\SmallSep % Used for spacing
}

\newcommand{\CVAbility}[3]{ % New command for the item descriptions
\textbf{\color{RoyalBlue} #1} \hfill #2 \par
#3
\SmallSep % Used for spacing
}

\newcommand{\CVSkill}[2]{ % New command for the item descriptions
\textbf{#1}\par
#2
\par
\SmallSep % Used for spacing
}

\newcommand{\bluebullet}{\textcolor{RoyalBlue}{$\circ$}~~} % New command for the blue bullets

%counters to calculate my age.
%Place `\theage\' in the document where you want your age to appear.
%year counter, month counter, and day counter
\newcounter{dobyear}\setcounter{dobyear}{1995}
\newcounter{dobmonth}\setcounter{dobmonth}{6}
\newcounter{dobday}\setcounter{dobday}{25}
%calculate number of years
\newcounter{age}\setcounter{age}{\the\year}\addtocounter{age}{-\thedobyear}
%condition to test for birthmonth, then birth day.
\ifnum\the\month>\thedobmonth\else%
	\ifnum\the\day<\thedobday\addtocounter{age}{-1}\fi\fi
